\documentclass[11pt]{article}
\usepackage{amsmath, amssymb, amsfonts}
\usepackage{float}
\parindent 0px
\pagestyle{empty}

\begin{document}

  %parenthesis, curly brackets, square brackets

  The distributive property states that $a(b+c)=ab+ac$, for all $a, b, c \in \mathbb{R}$. \\[6pt]
  The equivalence class of $a$ is $[a]$. \\[6pt]
  The set $A$ is defined to be $\{1, 2, 3\}$. \\[6pt]
  The movie ticket costs $\$11.50$.

  $$2\left(\frac{1}{x^2-{1}}\right)$$
  $$2\left[\frac{1}{x^2-{1}}\right]$$
  $$2\left\{\frac{1}{x^2-{1}}\right\}$$
  $$2\left\langle\frac{1}{x^2-{1}}\right\rangle$$
  $$2\left|\frac{1}{x^2-{1}}\right|$$

  $$\left.\frac{dy}{dx}\right|_{x=1}$$

  $$\left(\frac{1}{1+\left(\frac{1}{1+x}\right)}\right)$$


  Tables: \\

  % a table with 6 center-aligned columns (c)6
  % & is a cell data separator

  \begin{tabular}{|c||c|c|c|c|c|}
    \hline
    $x$ & 1 & 2 & 3 & 4 & 5 \\ \hline
    $f(x)$ & 10 & 11 & 12 & 13 & 14 \\ \hline
  \end{tabular}

  \vspace{1cm}

  \begin{table}[H]
    \centering
    % expands height in table cells
    \def\arraystretch{1.4}
    \begin{tabular}{|c||c|c|c|c|c|}
      \hline
      $x$ & 1 & 2 & 3 & 4 & 5 \\ \hline
      $f(x)$ & $\frac{1}{2}$ & 11 & 12 & 13 & 14 \\ \hline
    \end{tabular}
    \caption{These values represent the function $f(x)$}
  \end{table}

  \begin{table}[H]
    \centering
    \caption{The relationship between $f$ and $f'$.}
    % expands height in table cells
    \def\arraystretch{1.4}
    % content is left justified
    % make a (p) paragraph to wrap text around
    \begin{tabular}{|l|p{3in}|}
      \hline
      $f(x)$ & $f'(x)$ \\ \hline
      $x>0$ & The function $f(x)$ is increasing. The function $f(x)$ is increasing. 
      The function $f(x)$ is increasing. \\ \hline
    \end{tabular}
  \end{table}
  
  % in {align}, math mode is automatically triggered
  % in math mode, spaces are ignored unless forced (\,)
  % align equations on the = sign (&)
  % wrap in {align*} to turn off line numbers

  Arrays:
  \begin{align}
    5x^2\, \text{place your words here}\\
    5x^2-9&=x+3\\
    5x^2-x-12&=0\\
    &=12+x-5x^2
  \end{align}

  \begin{align*}
    5x^2-9&=x+3\\
    5x^2-x-12&=0\\
    &=12+x-5x^2
  \end{align*}

\end{document}
